\chapter[Introdução]{Introdução}
\label{chap:introducao}
	
	A procura por atendimento em busca de orientações e efetuação de matrícula em disciplinas por parte dos alunos no início do semestre, caracteriza-se como algo exaustivo. Esse fator está atrelado à grande demanda estabelecida pelos alunos.

	É possível contemplar formação de filas com grandes proporções em busca de um diálogo com o coordenador do curso. Adicionalmente, os coordenadores ficam sobrecarregados com tantos atendimentos que devem prestar. 
	
	De forma geral, o período de tempo destinado para atendimento no início do semestre configura-se como insuficiente. Além disso, os alunos não fazem nenhum agendamento prévio e assim, os coordenadores dos cursos de graduação não possuem uma estimativa de quantos alunos irão procurar pelo atendimento em um determinado dia.

	\section[Proposta]{Proposta}
	\label{sec:introducao_proposta}

		Uma proposta de solução para o problema retratado anteriormente, seria a criação de um aplicativo para plataforma móvel. Esse aplicativo seria desenvolvido levando em consideração as particularidades administrativas da universidade, contudo, as principais funcionalidades seriam:

		\begin{itemize}
			\item{Possibilitar ao aluno escolher o coordenador de sua preferência para atendimento;}
			\item{Solicitar ao aluno o motivo da busca pelo atendimento;}
			\item{Exibir ao aluno a documentação necessária para os serviços básicos e mais corriqueiros no ambiente universitário, bem como documentos que serão solicitados pelos coordenadores de forma mais frequente;}
			\item{Informar estatísticas da procura pelos tipos de serviços e motivos de busca pelo atendimento.}
		\end{itemize}

	\section[Desenvolvimento]{Desenvolvimento}
	\label{sec:introducao_desenvolvimento}

		A proposta é que o aplicativo seja projetado para ambiente mobile e para diferentes plataformas, atendendo os principais sistemas operacionais como Android, iOS e Windows Phone.

	\section[Usuários]{Usuários}
	\label{sec:introducao_usuarios}

		Esse aplicativo está voltado para todos os estudantes de graduação, que em geral, necessitam de atendimento por parte dos coordenadores no início do semestre.
		É importante ressaltar que a aplicação proposta também poderia estar sendo utilizada em outros momentos, contudo, destina-se aos períodos mais críticos.

	


