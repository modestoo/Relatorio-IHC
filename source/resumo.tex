\chapter[Resumo]{Resumo}
\label{chap:resumo}
	O presente relatório está centrado em uma problemática contemplada no Campus de Engenharia da Universidade de Brasília, localizado no Gama - Distrito Federal.

	A procura por atendimento em busca de orientações e efetuação de matrícula em disciplinas por parte dos alunos no início do semestre caracteriza-se como algo exaustivo. Esse fator está atrelado à grande demanda estabelecida pelos alunos. Adicionalmente, é importante ressaltar que há procura por atendimento no tocante aos assuntos de ênfases, estágios e dicas do mercado de trabalho.

	É possível contemplar formação de filas com grandes proporções em busca de um diálogo com o coordenador do curso. Adicionalmente, os coordenadores ficam sobrecarregados com tantos atendimentos que devem prestar. Dessa maneira, propõe-se a criação do aplicativo Fila Certa, de forma a sanar uma boa parcela do problema.

	\vspace{\onelineskip}
	   
	\noindent
	\textbf{Palavras-Chaves}: Fila Certa, Fila, Aplicativo.
