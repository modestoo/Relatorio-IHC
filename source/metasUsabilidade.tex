\chapter[Metas de Usabilidade]{Metas de Usabilidade}
	\label{sec:metasUsabilidade}

		\begin{itemize}
			\item{Ser eficaz no uso (eficácia);}
			\item{Ser eficiente no uso (eficiência);}
			\item{Ser segura no uso (segurança);}
			\item{Ser de boa utilidade (utilidade);}
			\item{Ser fácil de aprender (learnability);}
			\item{Ser fácil de lembrar como se usa (memorability).}
		\end{itemize}

	\section[Eficácia]{Eficácia}
	\label{sec:metasUsabilidade_eficacia}

		Como o sistema tem um propósito bem específico, e que quando acessado, é de interesse do usuário obter já informações de maneira objetiva, rápida e fácil, torna-se altamente eficiente o atendimento desses objetivos específicos. 

	\section[Eficiência]{Eficiência}
	\label{sec:metasUsabilidade_eficiencia}

		O sistema é autoexplicativo e prontamente configurado pois contém os elementos e componentes gráficos tais como imagens, símbolos e escritas que permitem ao usuário entender o seu funcionamento sem maiores complicações. Então, uma vez utilizado, os demais acessos se tornarão mais familiares ao atendimento das necessidades do usuário. 

	\section[Segurança]{Segurança}
	\label{sec:metasUsabilidade_eficiencia}

		O sistema não oferecerá margem para tornar a sua manipulação insegura, pois as funcionalidades terão um conjunto, ou domínio já determinado, como as opções que o usuário poderá utilizar como filtros para refinar as consultas. Os parâmetros de entrada serão fixos, ou seja, o usuário irá selecionar de acordo com o interesse no momento e o sistema irá processar de acordo com esses parâmetros, impedindo que anomalias ou erros graves ocorram por parte de entradas mal especificadas.

	\section[Utibilidade]{Utibilidade}
	\label{sec:metasUsabilidade_eficiencia}

		A especialidade do sistema é uma só: prover um mecanismo de otimização para agendamento de atendimentos por parte dos coordenadores de graduação, podendo esse agendamento ser filtrada por uma série de parâmetros de acordo com o interesse do usuário.

	\section[Capacidade de Aprendizagem]{Capacidade de Aprendizagem}
	\label{sec:metasUsabilidade_aprendizagem}

		O sistema é totalmente interativo, desde seu processo de instalação à sua manipulação, dependendo totalmente de ações do usuário para seu funcionamento. Como é evidente a crescente utilização de aplicações em smartphones, a utilização do sistema se torna básica, sem a necessidade de tutoriais, manuais, cursos ou vídeo aulas para sua manipulação das funcionalidades básicas. 

	\section[Capacidade de Memorização]{Capacidade de Memorização}
	\label{sec:metasUsabilidade_memorizacao}

		Como o sistema é intuitivo, amigável e possui uma usabilidade favorável à capacidade de aprendizagem, por possuir elementos gráficos como botões que direcionariam as escolhas, descrições de cada atividade exercida por cada setor a partir da seleção dos filtros, o histórico de aprendizagem é facilitado, possibilitando ao usuário a qualquer tempo lembrar-se do passo a passo até seu resultado final, atendendo à necessidade naquele momento. 
