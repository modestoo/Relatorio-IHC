\chapter[Requisitos Funcionais e Não Funcionais]{Requisitos Funcionais e Não Funcionais}
\label{chap:requisitos}
	
	Antigamente dizia-se que requisitos eram sinônimos de funções, ou seja, tudo que o software deveria fazer funcionalmente. No entanto, atualmente assumiu-se que requisitos de software é muito mais do que apenas funções. Requisitos são, além de funções, objetivos, propriedades, restrições que o sistema deve possuir para satisfazer contratos, padrões ou especificações de acordo com o(s) usuário(s). De forma mais geral um requisito é uma condição necessária para satisfazer um objetivo.

	Portanto, um requisito é um aspecto que o sistema proposto deve fazer ou uma restrição no desenvolvimento do sistema. Vale ressaltar que em ambos os casos devemos sempre contribuir para resolver os problemas do cliente e não o que o programador ou um arquiteto deseja. Dessa forma, o conjunto dos requisitos como um todo representa um acordo negociado entre todas as partes interessadas no sistema. Isso também não significa que o programador, arquiteto ou um analista bem entendido no assunto de tecnologia não possam contribuir com sugestões e propostas que levem em conta o desejo do cliente.

	\section[Requisitos Funcionais]{Requisitos Funcionais}
	\label{sec:requisitos_funcionais}

		Requisitos funcionais descrevem as funcionalidades que se espera que o sistema disponibilize, de uma forma completa e consistente. É aquilo que o utilizador espera que o sistema ofereça, atendendo aos propósitos para qual o sistema será desenvolvido.

		\begin{itemize}
			\item{\textbf{RF 01} - O sistema deve solicitar as informações de login para liberação da marcação de agendamentos de atendimento;}
			\item{\textbf{RF 02} - Em caso de primeiro uso, o sistema deverá possibilitar o cadastramento ao usuário, solicitando as seguintes informações: Nome, CPF, RG, Número de Matrícula, Curso e E-mail;}
			\item{\textbf{RF 03} - O sistema deve possiblitar que o aluno escolha o coordenador de sua preferência para atendimento;}
			\item{\textbf{RF 04} - O sistema deve solicitar ao aluno um resumo do motivo de busca pelo atendimento;}
			\item{\textbf{RF 05} - O sistema deve exibir a disponibilidade de horários do coordenador para um determinado dia escolhido pelo aluno;}
			\item{\textbf{RF 06} - O sistema deve possibilitar ao aluno informar sua chegada no dia e horário marcado para o atendimento;}
			\item{\textbf{RF 07} - O sistema deve disponibilizar uma opção para cancelamento de atendimento. Nesse caso, o cancelamento poderá ser feito com até 48 (quarenta e oito) horas de antecedência com relação ao horário marcado para atendimento;}
			\item{\textbf{RF 08} - O sistema deve possibilitar a visualização de estatísticas de atendimento, informando os motivos de busca mais corriqueiros.}
		\end{itemize}


	\section[Requisitos Não Funcionais]{Requisitos Não Funcionais}
	\label{sec:requisitos_nao_funcionais}

		Requisitos não funcionais referem-se a aspectos não-funcionais do sistema, como restrições nas quais o sistema deve operar ou propriedades emergentes do sistema. Costumam ser divididos em Requisitos não-funcionais de: Utilidade, Confiança, Desempenho, Suporte e Escalabilidade.

		\begin{itemize}
			\item{\textbf{RNF 01} - Execução do sistema nos sistemas operacionais Android, iOS e Windows Phone;}
			\item{\textbf{RNF 02} - O sistema não deverá levar mais do que 5 (cinco) segundos para retornar o processamento de algo solicitado pelo usuário.} 
		\end{itemize}


