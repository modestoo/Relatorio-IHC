\chapter[Avaliações dos Protótipos com os Usuários]{Avaliações dos Protótipos com os Usuários}
\label{chap:avaliacaoPrototipos}

	Antes de declarar um software pronto para uso, é importante saber se ele apóia adequadamente os usuários, nas suas tarefas e no ambiente em que será utilizado. Assim como testes de funcionalidade são necessários para se verificar a robustez da implementação, a avaliação de interface é necessária para se analisar a qualidade de uso de um software. Quanto mais cedo forem encontrados os problemas de interação ou de interface, menor o custo de se consertá-los.

	Um projetista não deve supor que basta seguir métodos e princípios de projeto de interfaces para garantir uma alta qualidade de uso de seu software. Além disto, também não deve presumir que os usuários são como ele próprio, e que portanto bastaria sua avaliação individual para atestar esta qualidade. Deve-se ter em mente que alguém vai avaliar a qualidade de uso do seu sistema, nem que seja apenas o usuário final.

	\section[Objetivo e Escopo das Avaliações]{Objetivo e Escopo das Avaliações}
	\label{sec:avaliacaoPrototipos_Objetivo}
		
		É importante ressaltar que existe uma grande variedade de produtos interativos. Esses produtos possuem uma vasta lista de características que necessitam ser avaliadas.
		
		A avaliação é necessária para a certificação de que os usuários podem utilizar o produto e apreciá-lo. De maneira geral, o objetivo da avaliação consiste em verificar como um design preenche as necessidades dos usuários.

		A seguir são apresentados alguns dos principais objetivos da realização das avaliações para o projeto.

		\begin{itemize}
			\item{Identificar as necessidades de usuários ou verificar o entendimento dos projetistas sobre estas necessidades;}
			\item{Identificar problemas de interação ou de interface;}
			\item{Investigar como uma interface afeta a forma de trabalhar dos usuários;}
			\item{Comparar alternativas de projeto de interface;}
			\item{Alcançar objetivos quantificáveis em métricas de usabilidade;}
			\item{Verificar conformidade com um padrão ou conjunto de heurísticas.}
		\end{itemize}


	\section[Primeira Avaliação com os Usuários - Protótipos de Papel]{Primeira Avaliação com os Usuários - \emph{Protótipos de Papel}}
	\label{sec:avaliacaoPrototipos_Primeira}

		Para realização da primeira avaliação do aplicativo Fila Certa, foi proposta a atividade de realização de um agendamento para atendimento junto a um determinado coordenador.

		\subsection[Características da Avaliação]{Características da Avaliação}
		\label{sec:primeiraAvaliacao_Caracteristicas}

			Para realização da avaliação, foi definido o Paradigma de Avaliação denominado como uma abordagem “Rápida e Suja”. A escolha esteve fortemente atrelada ao fato de haver um comportamento mais informal por parte dos usuários, bem como o fato dos avaliadores não precisarem apresentar demasiado controle.

			Quanto à quantidade de usuários, foi definida uma avaliação com 3 (três) usuários. As pessoas escolhidas para a avaliação são estudantes do Campus do Gama da Universidade de Brasília.

			A principal atividade para essa avaliação era o agendamento com um coordenador, logo, os usuários foram devidamente orientados para realizar a atividade de cadastrar um novo agendamento com um determinado coordenador de curso.

			Adicionalmente, foi aplicado um questionário com 5 (cinco) perguntas para recolhimento de informações e posterior análise. Esses dados são apresentados nas seções seguintes do documento.

		\subsection[Itens do Questionário Aplicados na Avaliação]{Itens do Questionário Aplicado na Avaliação}
		\label{sec:primeiraAvaliacao_Questionario}

			A seguir, tem-se uma exibição das perguntas que foram aplicadas para os usuários após realização da atividade sugerida no protótipo de papel.

			\begin{enumerate}
				\item{Você recomendaria este software para seus colegas?}
				\item{A forma como o sistema apresenta as informações é clara e compreensível?}
				\item{Você levou muito tempo para compreender e identificar as funcionalidades do software?}
				\item{A organização das funcionalidades pareceu consistente para você?}
				\item{Esse software é relevante para o seu contexto?}
			\end{enumerate}

			É importante ressaltar que as perguntas envolviam as seguintes opções de resposta: \textbf{Concordo}, \textbf{Discordo} e \textbf{Não sei}.

		\subsection[Resultados da Primeira Avaliação]{Resultados da Primeira avaliação}
		\label{sec:primeiraAvaliacao_Resultados}

			A seguir são apresentadas informações dos usuários participantes da avaliação, e posteriormente, os resultados da avaliação na visão de cada participante.

			1º Camila Caetano - Engenharia de Energia

			2º Jhéssica Isabel - Engenharia de Aeroespacial

			3º Jonathan Morais - Engenharia de Software

			\begin{table}[h]
				\centering 
				\begin{tabular}{|c|c|c|c|}

					\hline

					& Camila Caetano & Jhéssica Isabel & Jonathan Morais \\
					
					\hline
					
					Questão 1 & Concordo & Concordo & Concordo \\
					
					Questão 2 & Concordo & Concordo & Concordo \\
					
					Questão 3 & Discordo & Discordo & Não Sei \\

					Questão 4 & Concordo & Concordo & Concordo \\

					Questão 5 & Concordo & Concordo & Concordo \\

					\hline

				\end{tabular}
				\caption[Resultado da primeira avaliação do protótipo com usuários]{Resultado da primeira avaliação do protótipo com usuários.}
				\label{tab:primeiraAvaliacao_tables}
			\end{table}

		\subsection[Análise dos Resultados Obtidos para Primeira Avaliação]{Análise dos Resultados Obtidos para Primeira Avaliação}
		\label{sec:primeiraAvaliacao_Analise}

			De maneira geral, é possível contemplar uma boa reação por parte dos usuários. Durante a realização da avaliação do protótipo, todos os envolvidos chegaram a explicitar a aprovação que possuíam quanto à aplicação proposta.
		
			Contudo, o objetivo é refinar as funcionalidades pensadas para a aplicação, e adicionalmente, aumento de sofisticação quanto ao questionário e utilização de metodologias avançadas de análise de dados.

	\newpage
	\section[Segunda Avaliação com os Usuários - Protótipos de Baixa Fidelidade]{Segunda Avaliação com os Usuários - \emph{Protótipos de Baixa Fidelidade}}
	\label{sec:avaliacaoPrototipos_Segunda}

		Para a segunda avaliação resolveu-se manter a atividade proposta na primeira avaliação: realizar um agendamento com o coordenador do curso. Contudo, o protótipo passou por refinamentos, apresentando uma nova organização de interfaces.

		\subsection[Características da Avaliação]{Características da Avaliação}
		\label{sec:segundaAvaliacao_Caracteristicas}

			Para realização da segunda avaliação, manteve-se o Paradigma de Avaliação denominado como uma abordagem “Rápida e Suja”. Constatou-se que a abordagem e diálogo com os usuários seria algo informal e, adicionalmente, não seria necessário controle rigoroso por parte dos avaliadores.

			A principal atividade para essa segunda avaliação ainda era o agendamento com um coordenador, então, os usuários foram orientados para realizar a atividade de cadastrar um novo agendamento com um determinado coordenador de curso.

			Quanto à quantidade de usuários, manteve-se o valor da primeira avaliação, sendo que havia sido definida uma avaliação com 2 (dois) usuários. As pessoas escolhidas para a avaliação são estudantes do Campus do Gama da Universidade de Brasília.

			Adicionalmente, foi aplicado um questionário com 5 (cinco) perguntas para recolhimento de informações e posterior análise. Esses dados são apresentados nas seções seguintes do documento. Além dos aspectos citados anteriormente, a construção do questionário esteve baseada na abordagem do SUMI, no tocante às possibilidades de respostas e natureza dos questionamentos.

		\subsection[Itens do Questionário Aplicados na Avaliação]{Itens do Questionário Aplicado na Avaliação}
		\label{sec:segundaAvaliacao_Questionario}

			A seguir, tem-se uma exibição das perguntas que foram aplicadas para os usuários após realização da atividade sugerida no protótipo de baixa fidelidade.

			\begin{enumerate}
				\item{Você recomendaria este software para seus colegas?}
				\item{A forma como o sistema apresenta as informações é clara e compreensível?}
				\item{Você levou muito tempo para compreender e identificar as funcionalidades do software?}
				\item{A organização das funcionalidades pareceu consistente para você?}
				\item{Esse software é relevante para o seu contexto?}
			\end{enumerate}

			É importante ressaltar que as perguntas envolviam as seguintes opções de resposta: \textbf{Concordo}, \textbf{Discordo} e \textbf{Não sei}.

		\subsection[Resultados da Segunda Avaliação]{Resultados da Segunda avaliação}
		\label{sec:segundaAvaliacao_Resultados}

			A seguir são apresentadas informações dos usuários participantes da avaliação, e posteriormente, os resultados da avaliação na visão de cada participante.

			\begin{enumerate}
				\item{º Samantha de Oliveira Gil – 12/0135175 - \emph{Engenharia de Aeroespacial}}
				\item{º Ruan Nawe Herculanno Pereira – 13/0145173 – \emph{Engenharia de Software}}
			\end{enumerate}

			\begin{table}[h]
				\centering 
				\begin{tabular}{|c|c|c|}

					\hline

					& Samantha de Oliveira & Ruan Nawe \\
					
					\hline
					
					Questão 1 & Concordo & Concordo \\
					
					Questão 2 & Concordo & Concordo \\
					
					Questão 3 & Discordo & Discordo \\

					Questão 4 & Concordo & Concordo \\

					Questão 5 & Concordo & Concordo \\

					\hline

				\end{tabular}
				\caption[Resultado da segunda avaliação do protótipo com usuários]{Resultado da segunda avaliação do protótipo com usuários.}
				\label{tab:segundaAvaliacao_tables}
			\end{table}

			\textbf{Observações e Críticas dos Usuários}


			\begin{table}[h]
				\centering 
				\begin{tabular}{|p{13cm}|}

					\hline

					1º Samantha de Oliveira Gil – 12/0135175 -  \emph{Engenharia de Aeroespacial} \\
					
					\hline
					
					Ao exibir os resultados do agendamento, mudar o título “cadastro” para “confirmação” pois nessa parte eu demorei um pouco para perceber que era apenas uma pagina que exibe a minha escolha. Evitar escolher mais de três cores para o software. \\

					\hline

					2º Ruan Nawe Herculanno – 13/0145173 –  \emph{Engenharia de Software} \\
					
					\hline
					
					A logo marca não tá legal, poderia mudar. \\

					\hline

				\end{tabular}
				\caption[Observações e críticas dos usuários]{Observações e críticas dos usuários.}
				\label{tab:segundaAvaliacao_tables}
			\end{table}

		\subsection[Análise dos Resultados Obtidos para Segunda Avaliação]{Análise dos Resultados Obtidos para Segunda Avaliação}
		\label{sec:segundaAvaliacao_Analise}

			As críticas apresentadas servirão de base para as mudanças no próximo protótipo, que será de alta fidelidade. De maneira geral, foi possível contemplar uma boa aceitação por parte dos usuários que participaram da segunda avaliação.

	\newpage
	\section[Terceira Avaliação com os Usuários - Protótipos de Alta Fidelidade]{Terceira Avaliação com os Usuários - \emph{Protótipos de Alta Fidelidade}}
	\label{sec:avaliacaoPrototipos_Terceira}

		A terceira avaliação foi subdivida em duas interações, onde na segunda interação ocorerram aprimoramentos do protótipo segundo sugestões dos usuários avaliados na primeira interação.

		O paradigma de avaliação manteve-se, sendo utilizada a abordagem \emph{rápida e suja} com os usuários. Constatou-se que a abordagem e diálogo com os usuários seria algo informal e, adicionalmente, não seria necessário controle rigoroso por parte dos avaliadores.

		Para as interações foram definidas as seguintes atividades no qual os usuários deveria realizar no protótipo:

		\begin{enumerate}
			\item{Realizar um agendamento;}
			\item{Visualizar o último agendamento realizado;}
			\item{Cancelar o último agendamento realizado;}
			\item{\emph{Deslogar} do sistema.}
		\end{enumerate}

		Cabe ressaltar que para a avaliação o protótipo estava em ambiente \emph{mobile} sendo realizada no celular de um dos avaliadores.

		\subsection[1º Interação]{1º Interação}
		\label{sec:terceiraAvaliacao_1}

			A primeira interação foi realizada no dia 14 de novembro com a participação de 5 (cinco) usuários. As pessoas escolhidas para a avaliação são estudantes do Campus do Gama da Universidade de Brasilia.

			\subsubsection[Itens do Questionário Aplicados na 1º Interação]{Itens do Questionário Aplicado na 1º Interação}
			\label{sec:terceiraAvaliacao_1_Questionario}

				Com base no \textbf{SUMI}, levando em consideração o padrão de respostas e natureza de questões, foi definido um questionário com 15 afirmações para ser utilizado com o usuário:

				\begin{enumerate}
					\item{É muito exaustivo aprender a operar o software.}
					\item{Esse software demanda muito tempo para fornecer um feedback da requisição de entrada.}
					\item{As informações de auxílio fornecidas pelo software não são úteis.}
					\item{É difícil saber o que fazer em algumas telas do software.}
					\item{Utilizar este software é extremamente proveitoso.}
					\item{A documentação do software é bem fundamentada e útil.}
					\item{Sinto-me no controle quando utilizo este software.}
					\item{Sinto-me mais seguro em utilizar as funções mais familiares ao meu contexto e necessidade.}
					\item{As tarefas podem ser realizadas de uma maneira relativamente simples neste software.}
					\item{Este software não seria útil para o meu cotidiano.}
					\item{As necessidades dos usuários foram totalmente levadas em consideração para elaboração do software.}
					\item{São necessários muitos passos para concretização de uma tarefa no software.}
					\item{A organização dos menus se apresenta como algo lógico e consistente.}
					\item{O software apresenta uma interface gráfica extremamente atraente.}
					\item{Facilmente se esquece de como utilizar as funções do software.}
				\end{enumerate}

			\subsubsection[Resultados da 1º Interação]{Resultados da 1º Interação}
			\label{sec:terceiraAvaliacao_1_Resultados}

				A seguir são apresentadas informações dos usuários participantes da avaliação, e posteriormente, os resultados da avaliação na visão de cada participante.

				\begin{enumerate}
					\item{º \textbf{Usuário:} Ludimila S. Ferreira - \emph{Engenharia de Aeroespacial}}
					\item{º \textbf{Usuário:} Douglas M. Santos – \emph{Engenharia de Aeroespacial}}
					\item{º \textbf{Usuário:} Pedro de Lyra – \emph{Engenharia de Software}}
					\item{º \textbf{Usuário:} Letícia S. L. Barros – \emph{Engenharia de Aeroespacial}}
					\item{º \textbf{Usuário:} Fernanda Pimenta Cyrne – \emph{Engenharia de Eletrônica}}
				\end{enumerate}

				\begin{table}[h]
					\centering 
					\begin{tabular}{|c|c|c|c|c|c|}

						\hline

						Questão & Usuário 1 & Usuário 2 & Usuário 3 & Usuário 4 & Usuário 5\\
						
						\hline
						
						1 & Discordo & Discordo & Discordo & Discordo & Discordo \\
						
						2 & Discordo & Discordo & Discordo & Discordo & Discordo \\
						
						3 & Indeciso & Discordo & Discordo & Discordo & Discordo \\

						4 & Discordo & Discordo & Discordo & Discordo & Discordo \\

						5 & Concordo & Concordo & Concordo & Concordo & Concordo \\

						6 & Indeciso & Concordo & Indeciso & Concordo & Concordo \\

						7 & Discordo & Concordo & Concordo & Concordo & Concordo \\

						8 & Concordo & Concordo & Concordo & Concordo & Concordo \\

						9 & Concordo & Concordo & Concordo & Concordo & Concordo \\

						10 & Discordo & Discordo & Discordo & Discordo & Discordo \\

						11 & Discordo & Concordo & Concordo & Concordo & Concordo \\

						12 & Concordo & Discordo & Discordo & Discordo & Discordo \\

						13 & Concordo & Concordo & Concordo & Concordo & Concordo \\

						14 & Discordo & Concordo & Concordo & Concordo & Concordo \\

						15 & Discordo & Discordo & Discordo & Discordo & Discordo \\

						\hline

					\end{tabular}
					\caption[Resultado da 3º avaliação - 1º Interação do protótipo com usuários]{Resultado da 3º avaliação - 1º Interação do protótipo com usuários.}
					\label{tab:terceiraAvaliacao1_tables}
				\end{table}

			\subsubsection[Análise dos Resultados Obtidos para 1º Interação]{Análise dos Resultados Obtidos 1º Interação}
			\label{sec:terceiraAvaliacao_1_Analise}

				Para avaliação dos dados obtidos com a aplicação dos questionários, utilizou-se o Qui-Quadrado. No contexto de questionários de avaliação, especialmente baseado no SUMI, a análise é feita da seguinte maneira:

				\begin{itemize}
					\item{Para o questionário, define-se as questões relativas às metas de usabilidade;}
					\item{Define-se para cada pergunta qual o valor esperado de concordância, discordância e indecisão, conforme o protótipo definido;}
					\item{Aplica-se o questionário para obtenção dos valores observados de cada pergunta;}
					\item{Calcula-se  $X^2$;}
					\item{Logo após, verificar se o valor está dentro ou fora do valor 3,841 da tabela $Xc^2$;}
					\item{Dependendo do resultado, se estiver maior, rejeita-se a hipótese, se menor, aceita-se.}
				\end{itemize}

				A seguir, é possível contemplar três tabelas: uma possuindo a frequência observada (respostas fornecidas pelos usuários durante as avaliações); outra possuindo a frequência esperada (respostas esperadas pelos avaliadores) e mais uma, que possui os valores calculados pelo Método Qui-Quadrado.

				\begin{landscape}
				\begin{table}[h]
					\centering 
					\begin{tabular}{|c|c|c|c|c|c|c|c|c|c|}

						\hline
						& \multicolumn{3}{|c|}{\textbf{Frequência Observada}} & \multicolumn{3}{c|}{\textbf{Frequência Esperada}} & \multicolumn{3}{c|}{\textbf{Qui-Quadrado}} \\

						\hline

						Questão & Concordância & Discordância & Indecisão & Concordância & Discordância & Indecisão & Concordância & Discordância & Indecisão \\
						
						\hline
						
						1 & 0 & 5 & 0 & 1 & 3 & 1 & 1 & 1,3 & 1 \\
						
						2 & 0 & 5 & 0 & 1 & 3 & 1 & 1 & 1,3 & 1 \\
						
						3 & 0 & 4 & 1 & 1 & 3 & 1 & 1 & 0,3 & 0 \\

						4 & 0 & 5 & 0 & 1 & 3 & 1 & 1 & 1,3 & 1 \\

						5 & 5 & 0 & 0 & 3 & 1 & 1 & 1,3 & 1 & 1 \\

						6 & 3 & 0 & 2 & 2 & 1 & 2 & 0,5 & 1 & 0 \\

						7 & 4 & 1 & 0 & 3 & 1 & 1 & 0,3 & 0 & 1 \\

						8 & 5 & 0 & 0 & 3 & 1 & 1 & 1,3 & 1 & 1 \\

						9 & 5 & 0 & 0 & 3 & 1 & 1 & 1,3 & 1 & 1 \\

						10 & 0 & 5 & 0 & 1 & 3 & 1 & 1 & 1,3 & 1 \\

						11 & 4 & 1 & 0 & 3 & 1 & 1 & 0,3 & 0 & 1 \\

						12 & 1 & 4 & 0 & 1 & 3 & 1 & 0 & 0,3 & 1 \\

						13 & 5 & 0 & 0 & 3 & 1 & 1 & 1,3 & 1 & 1 \\

						14 & 4 & 1 & 0 & 3 & 1 & 1 & 0,3 & 0 & 1 \\

						15 & 0 & 5 & 0 & 1 & 3 & 1 & 1 & 1,3 & 1 \\

						\hline

					\end{tabular}
					\caption[Análise do Qui-Quadrado para avaliação da 1º Interação]{Análise do Qui-Quadrado para avaliação da 1º Interação.}
					\label{tab:terceiraAvaliacao2_tables}
				\end{table}

				\begin{flushleft}
				Para o cálculo do Qui-Quadrado, a seguinte relação foi utilizada:

				Onde:
				
				\textbf{o} – \emph{frequência observada}

				\textbf{e} – \emph{frequência esperada}

				Para o coeficiente do método Qui-Quadrado, é importante ressaltar que nenhum dos quesitos avaliados pelo questionário apresentaram necessidade de melhoria, uma vez que todos os valores estiveram abaixo do índice 3,841.
				\end{flushleft}
				\end{landscape}


		\subsection[2º Interação]{2º Interação}
		\label{sec:terceiraAvaliacao_2}

			A segunda interação foi realizada no dia 17 de novembro com a participação de 5 (cinco) usuários. As pessoas escolhidas para a avaliação são estudantes do Campus do Gama da Universidade de Brasilia.

			\subsubsection[Itens do Questionário Aplicados na 2º Interação]{Itens do Questionário Aplicado na 2º Interação}
			\label{sec:terceiraAvaliacao_2_Questionario}

				Para a segunda interação, não houve alteração na estrutura do questionário, permanecendo-se as mesmas afirmações listadas a seguir:

				\begin{enumerate}
					\item{É muito exaustivo aprender a operar o software.}
					\item{Esse software demanda muito tempo para fornecer um feedback da requisição de entrada.}
					\item{As informações de auxílio fornecidas pelo software não são úteis.}
					\item{É difícil saber o que fazer em algumas telas do software.}
					\item{Utilizar este software é extremamente proveitoso.}
					\item{A documentação do software é bem fundamentada e útil.}
					\item{Sinto-me no controle quando utilizo este software.}
					\item{Sinto-me mais seguro em utilizar as funções mais familiares ao meu contexto e necessidade.}
					\item{As tarefas podem ser realizadas de uma maneira relativamente simples neste software.}
					\item{Este software não seria útil para o meu cotidiano.}
					\item{As necessidades dos usuários foram totalmente levadas em consideração para elaboração do software.}
					\item{São necessários muitos passos para concretização de uma tarefa no software.}
					\item{A organização dos menus se apresenta como algo lógico e consistente.}
					\item{O software apresenta uma interface gráfica extremamente atraente.}
					\item{Facilmente se esquece de como utilizar as funções do software.}
				\end{enumerate}

			\subsubsection[Resultados da 2º Interação]{Resultados da 2º Interação}
			\label{sec:terceiraAvaliacao_2_Resultados}

				A seguir são apresentadas informações dos usuários participantes da avaliação, e posteriormente, os resultados da avaliação na visão de cada participante.

				\begin{enumerate}
					\item{º \textbf{Usuário:} Vanessa Andrade - \emph{Engenharia de Software}}
					\item{º \textbf{Usuário:} Rafael Fazzolino – \emph{Engenharia de Software}}
					\item{º \textbf{Usuário:} Hugo Martins – \emph{Engenharia de Software}}
					\item{º \textbf{Usuário:} Davi Gustavo – \emph{Engenharia de Aeroespacial}}
					\item{º \textbf{Usuário:} Emilie Morais Cyrne – \emph{Engenharia de Software}}
				\end{enumerate}

				\begin{table}[h]
					\centering 
					\begin{tabular}{|c|c|c|c|c|c|}

						\hline

						Questão & Usuário 1 & Usuário 2 & Usuário 3 & Usuário 4 & Usuário 5\\
						
						\hline
						
						1 & Discordo & Discordo & Discordo & Discordo & Discordo \\
						
						2 & Discordo & Discordo & Discordo & Discordo & Discordo \\
						
						3 & Discordo & Discordo & Discordo & Discordo & Discordo \\

						4 & Discordo & Discordo & Discordo & Discordo & Discordo \\

						5 & Concordo & Concordo & Concordo & Concordo & Concordo \\

						6 & Indeciso & Indeciso & Indeciso & Indeciso & Indeciso \\

						7 & Concordo & Concordo & Concordo & Concordo & Concordo \\

						8 & Concordo & Concordo & Concordo & Concordo & Concordo \\

						9 & Concordo & Concordo & Concordo & Concordo & Concordo \\

						10 & Discordo & Discordo & Discordo & Discordo & Discordo \\

						11 & Concordo & Concordo & Concordo & Concordo & Concordo \\

						12 & Discordo & Discordo & Discordo & Concordo & Discordo \\

						13 & Concordo & Concordo & Concordo & Concordo & Concordo \\

						14 & Indeciso & Indeciso & Concordo & Indeciso & Concordo \\

						15 & Discordo & Discordo & Discordo & Discordo & Discordo \\

						\hline

					\end{tabular}
					\caption[Resultado da 3º avaliação - 2º Interação do protótipo com usuários]{Resultado da 3º avaliação - 2º Interação do protótipo com usuários.}
					\label{tab:terceiraAvaliacao2_tables}
				\end{table}

			\subsubsection[Análise dos Resultados Obtidos para 2º Interação]{Análise dos Resultados Obtidos para 2º Interação}
			\label{sec:terceiraAvaliacao_2_Analise}

				Para avaliação dos dados do questionário aplicado na segunda interação da terceira avaliação, utilizou-se dos mesmos mecanismos da primeira interação.
				
				A seguir, é possível contemplar três tabelas: uma possuindo a frequência observada (respostas fornecidas pelos usuários durante as avaliações); outra possuindo a frequência esperada (respostas esperadas pelos avaliadores) e mais uma, que possui os valores calculados pelo Método Qui-Quadrado.


				\begin{landscape}
				\begin{table}[h]
					\centering 
					\begin{tabular}{|c|c|c|c|c|c|c|c|c|c|}

						\hline
						& \multicolumn{3}{|c|}{\textbf{Frequência Observada}} & \multicolumn{3}{c|}{\textbf{Frequência Esperada}} & \multicolumn{3}{c|}{\textbf{Qui-Quadrado}} \\

						\hline

						Questão & Concordância & Discordância & Indecisão & Concordância & Discordância & Indecisão & Concordância & Discordância & Indecisão \\
						
						\hline
						
						1 & 0 & 5 & 0 & 1 & 3 & 1 & 1 & 1,3 & 1 \\
						
						2 & 0 & 5 & 0 & 1 & 3 & 1 & 1 & 1,3 & 1 \\
						
						3 & 0 & 5 & 0 & 1 & 3 & 1 & 1 & 1,3 & 1 \\

						4 & 0 & 5 & 0 & 1 & 3 & 1 & 1 & 1,3 & 1 \\

						5 & 5 & 0 & 0 & 3 & 1 & 1 & 1,3 & 1 & 1 \\

						6 & 0 & 0 & 5 & 2 & 1 & 2 & 2 & 1 & 4,5 \\

						7 & 5 & 0 & 0 & 3 & 1 & 1 & 1,3 & 1 & 1 \\

						8 & 5 & 0 & 0 & 3 & 1 & 1 & 1,3 & 1 & 1 \\

						9 & 5 & 0 & 0 & 3 & 1 & 1 & 1,3 & 1 & 1 \\

						10 & 0 & 5 & 0 & 1 & 3 & 1 & 1 & 1,3 & 1 \\

						11 & 5 & 0 & 0 & 3 & 1 & 1 & 1,3 & 1 & 1 \\

						12 & 1 & 4 & 0 & 1 & 3 & 1 & 0 & 0,3 & 1 \\

						13 & 5 & 0 & 0 & 3 & 1 & 1 & 1,3 & 1 & 1 \\

						14 & 2 & 0 & 3 & 3 & 1 & 1 & 0,3 & 1 & 4 \\

						15 & 0 & 5 & 0 & 1 & 3 & 1 & 1 & 1,3 & 1 \\

						\hline

					\end{tabular}
					\caption[Análise do Qui-Quadrado para avaliação da 2º Interação]{Análise do Qui-Quadrado para avaliação da 2º Interação.}
					\label{tab:terceiraAvaliacao2_tables}
				\end{table}

				O protótipo de alta fidelidade passou por refinamentos após a primeira avaliação. Contudo, em dois quesitos, itens 6 e 14 do questionário, foram contemplados valores acima do índice 3,841 do Qui-Quadrado para as frequências analisadas da opção de resposta “Indecisão”.
				
				Adicionalmente, é importante ressaltar que quando a frequência observada excede a frequência cogitada pelos avaliadores no quesito de “Indecisão”, faz-se necessária a melhoria da apresentação do protótipo nesse aspecto avaliado. Porém, como esse resultado foi obtido na última interação do projeto, a melhoria seria efetuada em avaliações futuras.

				\end{landscape}