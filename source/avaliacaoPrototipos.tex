\chapter[Avaliações dos Protótipos com os Usuários]{Avaliações dos Protótipos com os Usuários}
\label{chap:avaliacaoPrototipos}

	Antes de declarar um software pronto para uso, é importante saber se ele apóia adequadamente os usuários, nas suas tarefas e no ambiente em que será utilizado. Assim como testes de funcionalidade são necessários para se verificar a robustez da implementação, a avaliação de interface é necessária para se analisar a qualidade de uso de um software. Quanto mais cedo forem encontrados os problemas de interação ou de interface, menor o custo de se consertá-los.

	Um projetista não deve supor que basta seguir métodos e princípios de projeto de interfaces para garantir uma alta qualidade de uso de seu software. Além disto, também não deve presumir que os usuários são como ele próprio, e que portanto bastaria sua avaliação individual para atestar esta qualidade. Deve-se ter em mente que alguém vai avaliar a qualidade de uso do seu sistema, nem que seja apenas o usuário final.

	\section[Objetivo e Escopo das Avaliações]{Objetivo e Escopo das Avaliações}
	\label{sec:avaliacaoPrototipos_Objetivo}
		
		É importante ressaltar que existe uma grande variedade de produtos interativos. Esses produtos possuem uma vasta lista de características que necessitam ser avaliadas.
		
		A avaliação é necessária para a certificação de que os usuários podem utilizar o produto e apreciá-lo. De maneira geral, o objetivo da avaliação consiste em verificar como um design preenche as necessidades dos usuários.

		A seguir são apresentados alguns dos principais objetivos da realização das avaliações para o projeto.

		\begin{itemize}
			\item{Identificar as necessidades de usuários ou verificar o entendimento dos projetistas sobre estas necessidades;}
			\item{Identificar problemas de interação ou de interface;}
			\item{Investigar como uma interface afeta a forma de trabalhar dos usuários;}
			\item{Comparar alternativas de projeto de interface;}
			\item{Alcançar objetivos quantificáveis em métricas de usabilidade;}
			\item{Verificar conformidade com um padrão ou conjunto de heurísticas.}
		\end{itemize}


	\section[Primeira Avaliação com os Usuários - Protótipos de Papel]{Primeira Avaliação com os Usuários - \emph{Protótipos de Papel}}
	\label{sec:avaliacaoPrototipos_Primeira}

		Para realização da primeira avaliação do aplicativo Fila Certa, foi proposta a atividade de realização de um agendamento para atendimento junto a um determinado coordenador.

		\subsection[Características da Avaliação]{Características da Avaliação}
		\label{sec:primeiraAvaliacao_Caracteristicas}

			Para realização da avaliação, foi definido o Paradigma de Avaliação denominado como uma abordagem “Rápida e Suja”. A escolha esteve fortemente atrelada ao fato de haver um comportamento mais informal por parte dos usuários, bem como o fato dos avaliadores não precisarem apresentar demasiado controle.

			Quanto à quantidade de usuários, foi definida uma avaliação com 3 (três) usuários. As pessoas escolhidas para a avaliação foram estudantes do Campus da Universidade de Brasília - Gama.

			Adicionalmente, foi aplicado um questionário com 5 (cinco) perguntas para recolhimento de informações e posterior análise. Esses dados são apresentados nas seções seguintes do documento.

		\subsection[Itens do Questionário Aplicados na Avaliação]{Itens do Questionário Aplicado na Avaliação}
		\label{sec:primeiraAvaliacao_Questionario}

			A seguir, tem-se uma exibição das perguntas que foram aplicadas para os usuários após realização da atividade sugerida no protótipo de papel.

			\begin{enumerate}
				\item{Você recomendaria este software para seus colegas?}
				\item{A forma como o sistema apresenta as informações é clara e compreensível?}
				\item{Você levou muito tempo para compreender e identificar as funcionalidades do software?}
				\item{A organização das funcionalidades pareceu consistente para você?}
				\item{Esse software é relevante para o seu contexto?}
			\end{enumerate}

			É importante ressaltar que as perguntas envolviam as seguintes opções de resposta: \textbf{Concordo}, \textbf{Discordo} e \textbf{Não sei}.

		\subsection[Resultados da Primeira Avaliação]{Resultados da Primeira avaliação}
		\label{sec:primeiraAvaliacao_Resultados}

			A seguir são apresentadas informações dos usuários participantes da avaliação, e posteriormente, os resultados da avaliação na visão de cada participante.

			1º Camila Caetano - Engenharia de Energia

			2º Jhéssica Isabel - Engenharia de Aeroespacial

			3º Jonathan Morais - Engenharia de Software

			\begin{table}[h]
				\centering 
				\begin{tabular}{|c|c|c|c|}

					\hline

					& Camila Caetano & Jhéssica Isabel & Jonathan Morais \\
					
					\hline
					
					Questão 1 & Concordo & Concordo & Concordo \\
					
					Questão 2 & Concordo & Concordo & Concordo \\
					
					Questão 3 & Discordo & Discordo & Não Sei \\

					Questão 4 & Concordo & Concordo & Concordo \\

					Questão 5 & Concordo & Concordo & Concordo \\

					\hline

				\end{tabular}
				\caption[Resultado da primeira avaliação do protótipo com usuários]{Resultado da primeira avaliação do protótipo com usuários.}
				\label{tab:primeiraAvaliacao_tables}
			\end{table}

		\subsection[Análise dos Resultados Obtidos para Primeira Avaliação]{Análise dos Resultados Obtidos para Primeira Avaliação}
		\label{sec:primeiraAvaliacao_Analise}

			De maneira geral, é possível contemplar uma boa reação por parte dos usuários. Durante a realização da avaliação do protótipo, todos os envolvidos chegaram a explicitar a aprovação que possuíam quanto à aplicação proposta.
		
			Contudo, o objetivo é refinar as funcionalidades pensadas para a aplicação, e adicionalmente, aumento de sofisticação quanto ao questionário e utilização de metodologias avançadas de análise de dados.


	\section[Segunda Avaliação com os Usuários - Protótipos de Baixa Fidelidade]{Segunda Avaliação com os Usuários - \emph{Protótipos de Baixa Fidelidade}}
	\label{sec:avaliacaoPrototipos_Segunda}

		Para a segunda avaliação resolveu-se manter a atividade proposta na primeira avaliação: realizar um agendamento com o coordenador do curso. Contudo, o protótipo passou por refinamentos, apresentando uma nova organização de interfaces.

		\subsection[Características da Avaliação]{Características da Avaliação}
		\label{sec:segundaAvaliacao_Caracteristicas}

			Para realização da segunda avaliação, manteve-se o Paradigma de Avaliação denominado como uma abordagem “Rápida e Suja”. Constatou-se que a abordagem e diálogo com os usuários seria algo informal e, adicionalmente, não seria necessário controle rigoroso por parte dos avaliadores.

			Quanto à quantidade de usuários, manteve-se o valor da primeira avaliação, sendo que havia sido definida uma avaliação com 2 (dois) usuários. As pessoas escolhidas para a avaliação foram estudantes do Campus da Universidade de Brasília - Gama.

			Adicionalmente, foi aplicado um questionário com 5 (cinco) perguntas para recolhimento de informações e posterior análise. Esses dados são apresentados nas seções seguintes do documento. Além dos aspectos citados anteriormente, a construção do questionário esteve baseada na abordagem do SUMI, no tocante às possibilidades de respostas e natureza dos questionamentos.

		\subsection[Itens do Questionário Aplicados na Avaliação]{Itens do Questionário Aplicado na Avaliação}
		\label{sec:segundaAvaliacao_Questionario}

			A seguir, tem-se uma exibição das perguntas que foram aplicadas para os usuários após realização da atividade sugerida no protótipo de baixa fidelidade.

			\begin{enumerate}
				\item{Você recomendaria este software para seus colegas?}
				\item{A forma como o sistema apresenta as informações é clara e compreensível?}
				\item{Você levou muito tempo para compreender e identificar as funcionalidades do software?}
				\item{A organização das funcionalidades pareceu consistente para você?}
				\item{Esse software é relevante para o seu contexto?}
			\end{enumerate}

			É importante ressaltar que as perguntas envolviam as seguintes opções de resposta: \textbf{Concordo}, \textbf{Discordo} e \textbf{Não sei}.

		\subsection[Resultados da Segunda Avaliação]{Resultados da Segunda avaliação}
		\label{sec:segundaAvaliacao_Resultados}

			A seguir são apresentadas informações dos usuários participantes da avaliação, e posteriormente, os resultados da avaliação na visão de cada participante.

			\begin{enumerate}
				\item{º Samantha de Oliveira Gil – 12/0135175 - \emph{Engenharia de Aeroespacial}}
				\item{º Ruan Nawe Herculanno Pereira – 13/0145173 – \emph{Engenharia de Software}}
			\end{enumerate}

			\begin{table}[h]
				\centering 
				\begin{tabular}{|c|c|c|}

					\hline

					& Samantha de Oliveira & Ruan Nawe \\
					
					\hline
					
					Questão 1 & Concordo & Concordo \\
					
					Questão 2 & Concordo & Concordo \\
					
					Questão 3 & Discordo & Discordo \\

					Questão 4 & Concordo & Concordo \\

					Questão 5 & Concordo & Concordo \\

					\hline

				\end{tabular}
				\caption[Resultado da segunda avaliação do protótipo com usuários]{Resultado da segunda avaliação do protótipo com usuários.}
				\label{tab:segundaAvaliacao_tables}
			\end{table}

			\textbf{Observações e Críticas dos Usuários}


			\begin{table}[h]
				\centering 
				\begin{tabular}{|p{13cm}|}

					\hline

					1º Samantha de Oliveira Gil – 12/0135175 -  \emph{Engenharia de Aeroespacial} \\
					
					\hline
					
					Ao exibir os resultados do agendamento, mudar o título “cadastro” para “confirmação” pois nessa parte eu demorei um pouco para perceber que era apenas uma pagina que exibe a minha escolha. Evitar escolher mais de três cores para o software. \\

					\hline

					2º Ruan Nawe Herculanno – 13/0145173 –  \emph{Engenharia de Software} \\
					
					\hline
					
					A logo marca não tá legal, poderia mudar. \\

					\hline

				\end{tabular}
				\caption[Observações e críticas dos usuários]{Observações e críticas dos usuários.}
				\label{tab:segundaAvaliacao_tables}
			\end{table}

		\subsection[Análise dos Resultados Obtidos para Segunda Avaliação]{Análise dos Resultados Obtidos para Segunda Avaliação}
		\label{sec:segundaAvaliacao_Analise}

			As críticas apresentadas servirão de base para as mudanças no próximo protótipo, que será de alta fidelidade. De maneira geral, foi possível contemplar uma boa aceitação por parte dos usuários que participaram da segunda avaliação.

	\section[Terceira Avaliação com os Usuários - Protótipos de Alta Fidelidade]{Terceira Avaliação com os Usuários - \emph{Protótipos de Alta Fidelidade}}
	\label{sec:avaliacaoPrototipos_Terceira}

		TO DO

		\subsection[Características da Avaliação]{Características da Avaliação}
		\label{sec:terceiraAvaliacao_Caracteristicas}

			TO DO

		\subsection[Itens do Questionário Aplicados na Avaliação]{Itens do Questionário Aplicado na Avaliação}
		\label{sec:terceiraAvaliacao_Questionario}

			TO DO

		\subsection[Resultados da Terceira Avaliação]{Resultados da Terceira avaliação}
		\label{sec:terceiraAvaliacao_Resultados}

			TO DO

		\subsection[Análise dos Resultados Obtidos para Terceira Avaliação]{Análise dos Resultados Obtidos para Terceira Avaliação}
		\label{sec:terceiraAvaliacao_Analise}

			TO DO






