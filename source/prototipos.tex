\chapter[Protótipos]{Protótipos}
\label{chap:prototipos}

	A seguir, são apresentados os prototipos idealizados a partir dos estudos efetuados durante o processo de estudo do problema.
	
	Serão descritos aqui protótipos de papel, protótipos de baixa fidelidade e de alta fidelidade, seguindo as fotografias das telas da respectiva solução de software. A proposta de solução como bem explicado nos capítulos anteriores trata-se de um \emph{app} (aplicativo móvel) para as plataformas Android, iOS e Windows Phone.

	\section[Protótipos de Papel]{Protótipos de Papel}
	\label{sec:prototipos_papel}

		A construção de protótipos em papel é uma técnica clássica de grande aceitação no meio dos especialistas em projetos de interfaces de usuário devido à sua simplicidade, ao seu baixo custo e por ser bastante efetiva. Ela consiste em esboçar telas e objetos de interação (de acordo com o projeto de interação proposto) em papéis no tamanho real esperado para cada um. Durante uma sessão de teste o esboço da janela principal é apresentado e é dada uma tarefa típica para ser executada pelo participante. Com um dedo o participante aponta e toca no esboço para indicar onde ele clicaria ou relata com que informação preencheria um particular campo de uma caixa de diálogo ou de um formulário eletrônico.

		Abaixo são apresentados as imagens dos protótipos de papel desenvolvidos para a aplicação.

		\newpage

		[imagens]

	\section[Protótipos de Baixa Fidelidade]{Protótipos de Baixa Fidelidade}
	\label{sec:prototipos_baixa}

		Os protótipos de baixa fidelidade, também chamados de rascunhos ou sketches, são concebidos ainda na fase inicial, durante a concepção do sistema.
		
		Desenhados geralmente à mão utilizando lápis, borracha e papel, essas representações são feitas de maneira rápida e superficial, apenas margeando a ideia do projeto e definindo superficialmente sua interação com o usuário, não se preocupando ainda com elementos de layout, cores, disposições, etc.

		Essa etapa é fundamental para a definição do produto e levantamento de requisitos.

		Abaixo são apresentados as imagens dos protótipos de papel desenvolvidos para a aplicação.

		\newpage

		[imagens]


	\section[Protótipos de Alta Fidelidade]{Protótipos de Alta Fidelidade}
	\label{sec:prototipos_alta}

		Os mockups ou protótipos funcionais constituem a representação mais próxima do sistema a ser desenvolvido. Em alguns casos, é possível simular o fluxo completo das funcionalidades, permitindo a interação do usuário como se fosse o produto final.

		A aparência visual, as formas de navegação e interatividade já são concebidas e aplicadas aos protótipos de alta fidelidade.

		Seu desenvolvimento é realizado na fase final de definição da interface, utilizando programas de design gráfico, como o Photoshop ou Fireworks; ferramentas de codificação front-end, como o Sublime Text ou Dreamweaver; e linguagens de programação front-end, como o HTML + CSS + jQuery.

		Abaixo são apresentados as imagens dos protótipos de papel desenvolvidos para a aplicação.

		\newpage

		[imagens]
